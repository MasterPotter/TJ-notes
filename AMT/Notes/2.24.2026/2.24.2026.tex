\documentclass{article}
\usepackage{graphicx, physics, dsfont}
\usepackage{fancyhdr}
\usepackage{hyperref}
\usepackage{ragged2e}
\usepackage{amsmath, amsthm}
\usepackage[margin=1in]{geometry}
\usepackage{setspace}
\usepackage{tikz}
\usetikzlibrary{positioning}
\pagestyle{fancy}
\lhead{Zoeb Izzi}
\rhead{2/9/2026}
\lfoot{Unit 1}
\rfoot{}
\begin{document}
\thispagestyle{fancy}
\begin{center}\LARGE{Advanced Mathematical Techniques (AMT) \\ Notes}\end{center}
\section{Today's stuff}
\subsection{Quiz Review}
Question 10:
What is the radius of convergence of the Maclaurin series converging to
\[\frac{1}{x^2+4x+5}\]
\[A: 1. B:. \sqrt2 C: 2 D: \sqrt5 E: 5. F: None. G: Function has no Maclaurin series.\]
\newline \newline Question 11:
Find the first three nonzero terms of the Maclauring expansion of the following function,
taking its value at x=0 as the one that makes it continuous there. What is the 
radius of convergence of this expansion?
\[\frac{x}{e^{2x}-e^x} = \frac{x}{1 + 2x + \frac{4x^2}{2!} + \frac{8x^3}{3!} + \cdots - (1 + x + \frac{x^2}{2!} + \frac{x^3}{3!}\cdots)}\]
\[ = \frac{1}{1 + \frac{3}{2}x + \frac{7}{6} x^2 + \cdots} = \frac{1}{1-(-\frac{3}{2}x - \frac{7}{6} x^2 - \cdots)}\]
\[ = 1 + \left(-\frac{3}{2}x - \frac{7}{6} x^2 - \cdots\right) + \left(\frac{3}{2}x + \frac{7}{6} x^2 + \cdots\right)^2 + \cdots\]
\[ = 1 - \frac{3}{2}x - \frac{7}{6} x^2 + \frac{9}{4}x^2 + \cdots\]
\[ = 1 - \frac{3}{2}x + \frac{13}{12} x^2 + \cdots\]
The series has a failure when:
\[e^{2x} = e^x\]
This happens when $2x = x$, i.e., $x = 0$. However, the series has a 1 in 
the denominator so it's all good there. Therefore only every $2\pi$ is 
problematic, so the ROC is $2\pi$.
\newline \newline
Question 15: Find the interval of convergence of:
\[\sum_{k=1}^{\infty} \frac{x^k}{2^k k}\]
Using the ratio test,
\[\lim_{k \to \infty} \left|\frac{a_{k+1}}{a_k}\right| = \lim_{k \to \infty} \left|\frac{x^{k+1}}{2^{k+1}(k+1)} \cdot \frac{2^k k}{x^k}\right| = \lim_{k \to \infty} \left|\frac{x k}{2(k+1)}\right| = \frac{|x|}{2}\]
The series converges when $\frac{|x|}{2} < 1$, i.e., $|x| < 2$. However, 
when we check the endpoints, we see that $x=-2$ make a convergent series and $x = 2$ 
make a divergent series. The interval of convergence is $(-2, 2)$.
\subsection{Dirichlet Series}
Remember that, by Euler:
\[\zeta(2) = \sum_{n=1}^{\infty} \frac{1}{n^2} = \frac{\pi^2}{6}\]
\[\zeta(4) = \sum_{n=1}^{\infty} \frac{1}{n^4} = \frac{\pi^4}{90}\]
\[\zeta(6) = \sum_{n=1}^{\infty} \frac{1}{n^6} = \frac{\pi^6}{945}\]
We would like to find the values of:
\[\beta(2) = \sum_{k=0}^{\infty} \frac{(-1)^k}{(2k+1)^2}\]
\[\lambda(2) = \sum_{k=0}^{\infty} \frac{1}{(2k+1)^2}\]
\[\eta(2) = \sum_{k=1}^{\infty} \frac{(-1)^{k+1}}{k^2}\]
Note that:
\[\beta(2) = 1 - \frac{1}{3^2} + \frac{1}{5^2} - \frac{1}{7^2} + \cdots\]
\[\lambda(2) = 1 + \frac{1}{3^2} + \frac{1}{5^2} + \frac{1}{7^2} + \cdots\]
\[\eta(2) = 1 - \frac{1}{2^2} + \frac{1}{3^2} - \frac{1}{4^2} + \cdots\]
Some of these are related to the zeta function, some are not. We see that
every single even number is divisible by 2, but not every single odd number
is divisible by a number, so we can't separate out every other odd number.
Thus $\beta(2)$ isn't related to $\zeta(2)$. However, it turns out that 
it's easy to get the odd $\beta(x)$ but not the evens. 
\[\lambda(2) = 1 + \frac{1}{3^2} + \frac{1}{5^2} + \frac{1}{7^2} + \cdots
= 1 + \frac{1}{2^2} + \frac{1}{3^2} + \frac{1}{4^2} + \cdots - \left(\frac{1}{2^2} + \frac{1}{4^2} + \frac{1}{6^2} + \cdots\right)\]
\[ = 1 + \frac{1}{2^2} + \frac{1}{3^2} + \frac{1}{4^2} + \cdots - \frac{1}{2^2}\left(1 + \frac{1}{2^2} + \frac{1}{3^2} + \cdots\right)\]
\[ = \frac{\pi^2}{6} - \frac{1}{4} \cdot \frac{\pi^2}{6} = \frac{\pi^2}{8}\]
We can use the same logic to find $\lambda(4)$:
\[\lambda(4) = 1 + \frac{1}{3^4} + \frac{1}{5^4} + \frac{1}{7^4} + \cdots
= 1 + \frac{1}{2^4} + \frac{1}{3^4} + \frac{1}{4^4} + \cdots - \left(\frac{1}{2^4} + \frac{1}{4^4} + \frac{1}{6^4} + \cdots\right)\]
\[ = 1 + \frac{1}{2^4} + \frac{1}{3^4} + \frac{1}{4^4} + \cdots - \frac{1}{2^4}\left(1 + \frac{1}{2^4} + \frac{1}{3^4} + \cdots\right)\]
\[ = \zeta(4) - \frac{1}{2^4}\zeta(4) = \frac{15}{16}\zeta(4) = \frac{15}{16}\cdot\frac{\pi^4}{90} = \frac{\pi^4}{96}\]
Now, let's find $\eta(2)$ and $\eta(4)$ with the same logic. $\lambda$ was just
$\zeta$ without the evens. $\eta$ is $\zeta$ without the evens but with alternating signs, so we can
just take $\lambda$ and alternate the signs. Thus:
\[\eta(2) = \left(1 - \frac{2}{2^2}\right)\zeta(2) = \left(1 - \frac{1}{2}
\right)\zeta(2) = \frac{1}{2}\zeta(2) = \frac{1}{2} \cdot \frac{\pi^2}{6} =
 \frac{\pi^2}{12}\]
\[\eta(4) = \left(1 - \frac{2}{2^4}\right)\zeta(4) = \left(1 - \frac{1}{8}
\right)\zeta(4) = \frac{7}{8}\zeta(4) = \frac{7}{8} \cdot \frac{\pi^4}{90}
 = \frac{7\pi^4}{720}\]
Basically, in terms of $\zeta$, we have:
\[\lambda(s) = \left(1 - \frac{1}{2^s}\right)\zeta(s)\]
\[\eta(s) = \left(1 - \frac{2}{2^s}\right)\zeta(s)\]
Let's go back to the sine thing we were doing a few days ago. Remember that:
\[\frac{\sin x}{x} = \prod_{n=1}^{\infty} \left(1 - \frac{x^2}{n^2 \pi^2}
\right)\]
This is messy, so let's make it a sum and take a log:
\[\ln\left(\frac{\sin x}{x}\right) = \sum_{n=1}^{\infty} \ln\left(1 - 
\frac{x^2}{n^2 \pi^2}\right)\]
Now, let's use the Maclaurin expansion:
\[-\sum_{n=1}^{\infty} \sum_{k=1}^{\infty}\frac{x^{2k}}{n^{2k} \pi^{2k}k} 
= -\sum_{k=1}^{\infty}\frac{x^{2k}}{\pi^{2k}k} \sum_{n=1}^{\infty} 
\frac{1}{n^{2k}} = -\sum_{k=1}^{\infty} \frac{\zeta(2k)}{\pi^{2k}} \cdot 
\frac{x^{2k}}{2k}\]
Taking the derivative of the left hand side gives us:
\[\cot x - \frac{1}{x} = -2 \sum_{k=1}^{\infty} \frac{\zeta(2k)}{\pi^{2k}}
\cdot x^{2k-1}\]

\subsection{Factorial function}
These will reconcile at the end. We can suppose that $n$ is a very large
natural number, and $x$ is moderately sized. Considering $(n+x)!$:
\[(n+x)! = (n+x)(n+x-1)(n+x-2)\cdots(n+1)n!\]
However, we can also write this as (take $n=300$, $x=2$ and this example
should make it easier to understand):
\[(x+n)! = (x+n)(x+n-1)(x+n-2)\cdots(x+1)x!\approx n^x \cdot n!\]
Also, 
\[1 \approx \frac{(n+x)(n-1+x)(n-2+x)\cdots(1+x)x!}{n^x \cdot n!}\]
Therefore, 
\[x! \approx \frac{n^x \cdot n!}{(n+x)(n-1+x)(n-2+x)\cdots(1+x)}\]
Notice that the last expression makes sense if $x$ is not an integer as well.
We can define that if x is a natural number (counting number):
\[\Gamma(x) := \lim_{n \to \infty} \frac{n^x \cdot n!}{(n+x)(n-1+x)(n-2+x)
\cdots(1+x)x}\]
Note that there is an extra $x$ in the denominator. Consider that this is 
known as the shifted factorial. Let's evaluate $\Gamma(1)$:
\[\Gamma(1) = \lim_{n \to \infty} \frac{n^1 \cdot n!}{(n+1)(n-1+1)(n-2+1)\cdots(1+1) \cdot 1} = \lim_{n \to \infty} \frac{n^1 \cdot n!}{(n+1) \cdot n!} = 1\]
\[\Gamma(x+1) = \lim_{n \to \infty} \frac{n^{x+1} \cdot n!}{(n+x+1)(n-1+x+1)
(n-2+x+1)\cdots(1+x+1)(x+1)}\]
\[ = \lim_{n \to \infty} \frac{n}{x+1+n} \cdot 
\frac{n^x \cdot n!}{(n+x)(n-1+x)(n-2+x)\cdots(1+x)x} = x\Gamma(x)\]
This tells us the Recurrence Identity:
\[\Gamma(x+1) = x\Gamma(x)\]
Since we know $\Gamma(1) = 1$:
\[\Gamma(2) = 1 \cdot \Gamma(1) = 1\]
\[\Gamma(3) = 2 \cdot \Gamma(2) = 2 \cdot 1 = 2\]
\[\Gamma(4) = 3 \cdot \Gamma(3) = 3 \cdot 2 = 6\]
In general, if $m \in {N}$, then \[\Gamma(m+1) = m!\].
Further, 
\[\frac{1}{\Gamma(x)} = \lim_{n\to\infty}\frac{n^x \cdot n!}{(n+x)(n-1+x)(n-2+x)\cdots(1+x)x}\]
\[\frac{1}{x\Gamma(x)} = \lim_{n\to\infty}\frac{n^x \cdot n!}{(n+x)(n-1+x)(n-2+x)\cdots(1+x)}
 = n^{-x} \cdot \frac{x+n}{n} \cdot \frac{x+n-1}{n-1} \cdots \frac{1+x}{1}\]
 \[ = \frac{1}{\Gamma(x+1)} = \lim_{n\to\infty} n^{-x} \cdot \prod_{k=1}^{n} \frac{x+k}{k} = \lim_{n\to\infty} \frac{1}{n^x}\prod_{k=1}^{n} \frac{x+k}{k}\]
We know the element inside the product diverges to $\infty$ and outside diverges
to $0$, so we have an indeterminate form. We need a function that behaves
like $\left( 1 - \frac x k\right)$ and will give us a difference of squares. That'll help us 
out with this simplification, and we'll divide by it anyway. Since we're dividing, let's 
make sure that it never equals 0. Let's use $e^{\frac x k}$. This has a 
Maclaurin expansion of:
\[e^{\frac x k} = 1 + \frac{x}{k} + \frac{x^2}{2k^2} + \cdots\]
In the limit, we see that this behaves as we need. Inserting a negative
to make it do the difference of squares, 
\[e^{-\frac x k} = 1 - \frac{x}{k} + \frac{x^2}{2k^2} - \cdots\]
Using this, 
\[\frac{1}{\Gamma(x)} = \lim_{n\to\infty} \left[\frac{1}{n^x} e^{xH_n}\prod_{k=1}^{n} \frac{x+k}{k} e^{-\frac{x}{k}}\right]= 
\lim_{n\to\infty} \left[ e^{-x \ln n} \cdot e^{xH_n} \prod_{k=1}^{n} e^{-\frac{x}{k}}\left( 1 + \frac{x}{k}\right)\right]\]
\[ = \lim_{n\to \infty} e^{x(H_n - \ln n)} \cdot \frac{1}{\prod_{k=1}^{n} e^{-\frac{x}{k}}\left( 1 + \frac{x}{k}\right)} = 
\lim_{n\to \infty} e^{x(H_n - \ln n)} \cdot \prod_{k=1}^{n} {e^{-\frac{x}{k}}}\left(1 + \frac{x}{k}\right)\]
Behaviorally, we can replace this with:
\[a_k = \left( 1 + \frac{x}{k}\right)e^{-\frac{x}{k}} -1\]
Recall that $e^{-\frac x k} = 1 - \frac{x}{k} + \frac{x^2}{2k^2} - \cdots$.
When the terms get very large, the only thing that will really matter is 
the first two terms, $1 - \frac{x}{k}$.
We can now use the limit comparison test on $a_k$ and $\frac{x^2}{k^2}$
\[\lim_{k\to \infty} \frac{\left( 1 + \frac{x}{k}\right)e^{-\frac{x}{k}} -1}{\frac{x^2}{k^2}} = \lim_{k\to \infty} \frac{\left( 1 + \frac{x}{k}\right)\left(1 - \frac{x}{k} + \frac{x^2}{2k^2} - \cdots\right) -1}{\frac{x^2}{k^2}} = \lim_{k\to \infty} \frac{-\frac{x^2}{2k^2} + O(k^{-3})}{\frac{x^2}{k^2}} = -\frac{1}{2}\]
This means that our infinite product will be fine all places except for at 0.
This gives us the infinite product representation of the Gamma function.
\[\frac{1}{\Gamma(x)} = e^{\gamma x} \prod_{k=1}^{\infty} \left(1 + \frac{x}{k}\right)e^{-\frac{x}{k}}\]
with $\frac{1}{\Gamma(x+1)}$ defined throughout the whole complex plane and 
$\Gamma(x)$ is defined when $x$ is a positive integer.

Now, linking back to the previous section:
\[\frac{1}{\Gamma(1+x)} \cdot \frac{1}{\Gamma(1-x)} = 
e^{\gamma x} \prod_{k=1}^{\infty} \left(1 + \frac{x}{k}\right)e^{-\frac{x}{k}} \cdot e^{\gamma(-x)} \prod_{j=1}^{\infty} \left(1 + \frac{-x}{j}\right)e^{-\frac{-x}{j}}\]
\[\frac{1}{x\Gamma(x)\Gamma(1-x)} = \prod_{k=1}^{\infty} \left(1 + \frac{x}{k}\right)\left(1 - \frac{x}{k}\right) = \prod_{k=1}^{\infty} \left(1 - \frac{x^2}{k^2}\right) = \frac{\sin \pi x}{\pi x}\]
giving us the reflection identity for the Gamma function:
\[\Gamma(x)\Gamma(1-x) = \frac{\pi}{\sin \pi x}\]

Note a special value of the Gamma function:
\[\Gamma(\frac 1 2)\Gamma(1-\frac 1 2) = \Gamma^2\left(\frac{1}{2}\right)
 = {\pi}\]
 \[\Gamma\left( \frac{1}{2} \right) = \sqrt{\pi}\]
Further, note that:
\[\ln \frac{1}{\Gamma(x+1)} = -\ln\Gamma(x+1) = \gamma x + \sum_{k=1}^{\infty}
\left[ \ln\left( 1 + \frac x k\right) - \frac x k\right]\]
\[\ln \Gamma(1+x) = -\gamma x + \sum_{k=1}^{\infty}\left[ \frac{\left(-\frac{x}{k}\right)^j}{j} + \frac x k\right]\]
\[ = -\gamma x + \sum_{k=1}^{\infty} \sum_{j=1}^{\infty} \frac{(-1)^j}{j} \cdot \frac{x^j}{k^j}\]
\[ = -\gamma x + \sum_{j=2}^{\infty} \frac{(-1)^j \zeta(j)}{j} x^j\]
when $|x| < 1$.
\end{document}