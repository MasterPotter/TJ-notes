\documentclass{article}
\usepackage{graphicx, physics, dsfont}
\usepackage{fancyhdr}
\usepackage{hyperref}
\usepackage{ragged2e}
\usepackage{amsmath, amsthm}
\usepackage[margin=1in]{geometry}
\usepackage{setspace}
\usepackage{tikz}
\usetikzlibrary{positioning}
\pagestyle{fancy}
\lhead{Zoeb Izzi}
\rhead{2/9/2026}
\lfoot{Unit 1}
\rfoot{}
\begin{document}
\thispagestyle{fancy}
\begin{center}\LARGE{Advanced Mathematical Techniques (AMT) \\ Notes}\end{center}
\section{Today's stuff}
\subsection{Intro}
We know that this works for absolute convergence:
    \[\prod_{k=1}^{\infty} \text{ converges absolutely if and only if }
     \sum_{k=1}^{\infty}a_k \text{ does.}\]
Also, recall that:
\[\frac12 |a_k| < \left|\ln(1 + a_k)\right| < 2|a_k|\]
which is true for all values of $a_k$ less than 0.1. It works for larger ones too,
but we are always considering the limit as $a_k \to 0$, so it doesn't make a huge
difference, which one we pick. It's always about the 3-way inequality, like the 
limit comparison test, root test, ratio test, etc. 
\subsection{Examples of Infinite Products}
Write the polynomial of minimal degree that is 5 when $x=0$ and has roots 
at $-4,-3,2,5$.
\begin{align*}
    P(x) &= a(x+4)(x+3)(x-2)(x-5) \\
    P(0) &= 5 \\
    5 &= a(4)(3)(-2)(-5) \\
    5 &= 120a \\
    a &= \frac{1}{24} \\
    P(x) &= \frac{1}{24}(x+4)(x+3)(x-2)(x-5)\\
    &= \left( 1 + \frac{x}{4}\right)\left( 1 + \frac{x}{3}\right)\left( 1 - \frac{x}{2}\right)
    \left( 1 - \frac{x}{5}\right)
\end{align*}
which looks eerily similar to the what we wrote above.
\newline\newline
Consider a new example:
\[\sin x\]
However, if we tried the above trick, we'd end up with 0 as our polynomial. 
Let's try, instead, dividing by $x$ for now:
\[\frac{\sin x}{x} = \left( 1 + \frac{x}{\pi}\right)\left( 1 - \frac{x}{\pi}\right)
\left( 1 + \frac{x}{2\pi}\right)\left( 1 - \frac{x}{2\pi}\right) \cdots\]
which is equal to:
\[\prod_{k=1}^{\infty} \left( 1 + \frac{x}{k\pi}\right)\prod_{j=1}^{\infty}\left( 1 - 
\frac{x}{j\pi}\right)\]
This appears to be a product of two divergent infinite products, but we can see that 
they diverge to the product of $0 \cdot \infty$, which is indeterminate. We can try
to use difference of squares to get:
\[\prod_{k=1}^{\infty} \left( 1 - \frac{x^2}{k^2\pi^2}\right)\]
Now, by comparison to the p-series with $p=2$, we know this converges. This is called the 
\textbf{infinite product representation} for $\frac{\sin x}{x}$. There are a bunch: Taylor, Maclaurin,
Product, etc. The Taylor/Maclaurin series give us the derivatives easily, while the 
product representation gives us the roots easily.
\newline\newline
We still need to remove the $x$ in the denominator. Let's try this:
\[\left(1 - \frac{x^2}{\pi^2}\right)\left(1 - \frac{x^2}{2^2\pi^2}\right)\left(1 - \frac{x^2}{3^2\pi^2}\right) \cdots\]
Combining them all together and grouping by term, we get:
\[1 - \frac{x^2}{\pi^2}\left(1 + \frac{1}{2^2} + \frac{1}{3^2} + \cdots\right) +\cdots
= 1 - \frac{x^2}{\pi^2}\sum_{k=1}^{\infty} \frac{1}{k^2} + \frac{x^4}{\pi^4} 
\sum_{k=1}^{\infty} \sum_{j=k+1}^{\infty} \frac{1}{k^2j^2} + \cdots\]
Now, we know that this must be the same as the Maclaurin expansion for $\frac{\sin x}{x}$.
This tells us that:
\[1 - \frac{x^2}{6} + \frac{x^4}{120} - \cdots = 1 - \frac{x^2}{\pi^2}\sum_{k=1}^{\infty} \frac{1}{k^2} + \frac{x^4}{\pi^4} 
\sum_{k=1}^{\infty} \sum_{j=k+1}^{\infty} \frac{1}{k^2j^2} + \cdots\]
This means that:
\[\frac16 = \frac{1}{\pi^2}\sum_{k=1}^{\infty} \frac{1}{k^2}\]
which tells us that:
\[\sum_{k=1}^{\infty} \frac{1}{k^2} = \frac{\pi^2}{6}\]
However, it also tell us that:
\[\frac{1}{120} = \frac{1}{\pi^4}\sum_{k=1}^{\infty} \sum_{j=k+1}^{\infty} \frac{1}{k^2j^2}
 = \left(\sum_{k=1}^{\infty} \frac{1}{k^2}\right)^2 = \left(\frac{\pi^2}{6}\right)^2\]
 \[ = \sum_{k=1}^{\infty} \frac{1}{k^4} + 2\sum_{k=1}^{\infty} \sum_{j=k+1}^{\infty} \frac{1}{k^2j^2}\]
similar to the classic $(a+b)^2 = a^2 + 2ab + b^2$. Plugging in, we get:
\[\frac{\pi^4}{36} = \sum_{k=1}^{\infty} \frac{1}{k^4} + 2\times \frac{\pi^2}{120}
= \frac{\pi^4}{6} \cdot \frac{1}{15} = \frac{\pi^4}{90}\]
Notice that we don't have this done for any of the odd numbers. That's because the 
odd numbers aren't going to be present in the even function that we did via the difference
of squares. If we tried to rip apart the difference of squares, we see that we would get an 
indeterminate form, which is no bueno. 
A related series is the Riemann zeta function:
\[\zeta(s) = \sum_{n=1}^{\infty} \frac{1}{n^s}\]
Another is the set of Dirichlet functions:
\[\eta(s) = \sum_{n=1}^{\infty} \frac{(-1)^{n-1}}{n^s} = 1 - \frac{1}{2^s} + \frac{1}{3^s} - \frac{1}{4^s} + \cdots\]
\[\lambda(s) = \sum_{n=1}^{\infty} \frac{1}{(2n-1)^s} = 1 + \frac{1}{3^s} + \frac{1}{5^s} + \frac{1}{7^s} + \cdots\]
\[\beta(s) = \sum_{n=1}^{\infty} \frac{(-1)^{n-1}}{(2n-1)^s} = 1 - \frac{1}{3^s} + \frac{1}{5^s} - \frac{1}{7^s} + \cdots\]
 \end{document}