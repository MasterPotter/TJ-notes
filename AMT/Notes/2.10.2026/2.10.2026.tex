\documentclass{article}
\usepackage{graphicx, physics, dsfont}
\usepackage{graphicx}
\usepackage{fancyhdr}
\usepackage{hyperref}
\usepackage{ragged2e}
\usepackage{amsmath, amsthm}
\usepackage[margin=1in]{geometry}
\usepackage{setspace}
\usepackage{tikz}
\usetikzlibrary{positioning}
\usepackage[
]{geometry}
\pagestyle{fancy}
\lhead{Zoeb Izzi}
\rhead{2/10/2026}
\lfoot{Unit 1}
\rfoot{}
\begin{document}
\thispagestyle{fancy}
\begin{center}\LARGE{Advanced Mathematical Techniques (AMT) \\ Notes}\end{center}
\section{Recap}
We can use the integral comparison test to estimate the sums of a finite number of terms for divergent series. We can use that 
concept to consider a different combination, and a convergent series. 
\section{Today's stuff}
\subsection{Integral Test}
Consider a sequence $S_n = H_n - \ln(n+1)$. Analyzing the graph of $y = \frac1x$, let's make some rectangles with the upper left
point lying on the graph and width 1. Considering these rectangles going from $x=1$ to $x=n$, the sum of the areas of all
of the boxes is $1 + \frac12 + \frac13 + \dots + \frac1n$.
\[H_n = \sum_{k=1}^n \frac1k\]
Now, the area under the curve is:
\[\int_{1}^{n+1}\frac1x dx = \ln(n+1)\]
Let's look at the differences between the box area and what we'd call the integral area, which is the $S_n$ series. $S_1$
describes the difference between the first rectangle and the integral over that boundary, $S_2$ over the first two, etc.
From this, since that area is strictly positive (approaching 0), we know that $S_n$ is strictly increasing. Further, by looking
at the graph, we can sort of replicate the error terms of all of the subsequent terms in the first bar by drawing the curve there
and all of the boxes are doomed to fit, since math. It's like if you just shifted the error part of each bar sideways until it 
clicked into place in the first bar.\newline \newline
Now, because they all fit in the first bar, we know that the sum of all the errors must be less than the area of the first box,
which we already established was 1. Therefore, the sequence $S_n$ is bounded by 1, and since it's monotonic and bounded,
it \textbf{MUST} converge. Let's try and make an estimate for the value, though.\newline \newline
Remember from yesterday, 
\[28.162 < H_{10^{12}} < 28.2574\]
Therefore, 
\[28.162 - \ln(10^{12}+1) < H_{10^{12}} - \ln(10^{12}+1) < 28.2574 - \ln(10^{12}+1)\]
\[0.531 < H_{10^{12}} - \ln(10^{12}+1) < 0.6264\]
The average of the bounds is around $0.5772$, which is very important - so much so that it has it's own variable, 
$\gamma$.
\[\gamma = \lim_{n\xrightarrow{}\infty} (H_n - \ln n)\]
Don't worry, the $n+1$ vs $n$ doesn't make a difference. Back to work - let's recall the power series, where:
\[\sum_{n=1}^{\infty} \frac{1}{n^p}\]
where if $p>1$, it converges, and otherwise it diverges. We can use the integral comparison test to estimate the value
of this convergent series, such as the $p$-series with $p=3$.
\[\sum_{n=1}^{10}\frac1{n^3} = 1.19753\]
\[\int_{11}^{\infty} \frac1{n^3}< \sum_{n=11}^{\infty} \frac1{n^3}< \int_{10}^{\infty}\frac1{n^3}\]
\[1.2017< \sum_{n=1}^{\infty} \frac1{n^3} < 1.2025\]
We can use the integral comparison test to see how much each term contributes. Also, as a fun fact, the error is about
$\frac1{11^3}$, and 11 was the first term we didn't compute. 
\subsection{Root Test}
The ratio test (like the root test) is basically a comparison test for geometric series. It basically asks the question, 
"If you were a geometric series, which would you be?"
Suppose the following:
\[ \lim_{n\xrightarrow{}\infty} \sqrt[n]{a_n} = r \]
Then, given any $\varepsilon > 0 \in N$ such that $r - \varepsilon < \sqrt[n]{a_n} < r + \varepsilon \hspace{1mm} \forall
\hspace{1mm} n > N$.
\[(r-\varepsilon)^n < a_n < (r+\varepsilon)^n\]
\[\sum_{n=N+1}^{N+P}(r-\varepsilon)^n < \sum_{n=N+1}^{N+P} a_n < \sum_{n=N+1}^{N+P} (r+\varepsilon)^n\]
Note that if $r < 1$, we then we can choose $\varepsilon$ so small that $r + \epsilon < 1$ as well. This is a strict limit - 
r \textbf{MUST} be less than 1. It must be a fixed number such that $r < 1$, because we need to be able to add a little bit 
to it - just a small amount - and it still must be less than 1. In other words, I can take $\varepsilon = \frac{1-\varepsilon}{2}$
and, as long as $r < 1$, we have satisfied this. Since this means that the last series in that inequality 
converges and so does $a_n$, and conversely, if $r > 1$, we know that the last series in that inequality diverges, and 
thus so does $a_n$.
\newline \newline
However, if $r = 1$, then no matter how small $\varepsilon$ is, we know that the first series in that inequality converges
and the last one diverges, meaning that we know nothing about $a_n$ and the test is inconclusive. Note that you can't really use 
the root test unless $k$ or whatever the main variable is appears in the power.
\subsection{Power series}
A power series is a series of the form:
\[\sum_{n=0}^{\infty} c_n(z-z_0)^n\]
where $z_0$ is called the "center" of the series. This series and the root test were made for each other. This series converges
when:
\[\lim_{n\rightarrow\infty} \sqrt[n]{|c_n||z-z_0|^n} < 1\] 
The goal is to get the root and the power to agree, because that is what collapses the inner calculation. Anyways,
\[\lim_{n\rightarrow\infty} |z-z_0|\sqrt[n]{|c_n|} < 1\]
We define the radius of convergence $R$ such that: 
\[\frac1R = \lim_{n\rightarrow\infty} \sqrt[n]{|c_n|}\]
The power series converges on $|z-z_0| < R$. We used $z$ on purpose because nothing requires $x$ or $z$ to be real. If we enter
the complex plane, we can use the radius of convergence to determine the circle centered around $z_0$ such that, if $z$ is in 
that circle, the power series converges. Quite a step up from the old radius of convergence which dealt with purely real 
numbers. The thing is the power series can only converge on a disk in the complex plane, and that comes as a result of the root test.
The radius of this disk is the distance from the center to the \textbf{closest} "problem point" in the function. We define 
a "problem point" as a place where you're dividing by 0, taking a logarithm of 0, taking a non-natural power of 0, or doing
some other thing that you're not supposed to do.\newline \newline Example: Find the power series centered at 3 that converges to 
$\frac{2}{5-z}$.
\[\frac{2}{5-z} = \frac{2}{5-(z-3+3)} = \frac{2}{2-(z-3)} = \frac1{1-\frac{z-3}{2}} = \sum_{n=0}^{\infty}\left( \frac{z-3}2\right)^n
 = \sum_{n=0}^{\infty}\left( \frac{1}{2}\right)^n (z-3)^n\]
 Moreover,
 \[\frac1R = \lim_{n\rightarrow\infty} \sqrt[n]{0.5^n} = 0.5 = \frac12\]
 and the distance from the center to the nearest problem point (5) is indeed 2.
 Therefore, $|z-3|<2$.\newline \newline
 Another practice problem is to do the same but centered at -1.
 \[\frac2{5-z} = \frac2{6-(z+1)} = \frac{\frac13}{1-\frac{z+1}{6}} = \frac13\sum_{n=0}^{\infty} \left( \frac{z+1}6\right)^n\]
which will lead to a radius of convergence of 6, as expected. \newline\newline
New practice problem: Find the first four nonzero terms in the Maclaurin expansion of $\frac2{3-2x+x^2}$ (centered at 0).
\[\frac2{3-2x+x^2} = \frac2{2+(1-x)^2} = \frac1{1+\frac{(x-1)^2}{2}} = \sum_{n=0}^{\infty}\left[ -\frac{(x-1)^2}{2}\right]^n\]
Let's write out some terms now.
\[1 - \frac{(x-1)^2}{2} + \frac{(x-1)^4}{2^2} - ... \]
However, this isn't centered at 0, rather it's centered at 1. Also, if you wanted to consolidate the terms you need, you end up 
with a lot of uncontrolled expansion, and a lot of infinities. Let's regain control and try it some other way.
\[\frac2{3-2x+x^2} = \frac{\frac23}{1-\frac{2x-x^2}{3}} = \frac23 \sum_{n=0}^{\infty}\left(\frac{2x-x^2}{3}\right)^n\].
THAT IS NOT A POWER SERIES! It does not have to converge in a disk! This series in particular DOES NOT converge on a disk. 
However, the Maclaurin series is a power series and it will converge in a disk. 
\[\frac23 \left[ 1 + \frac{2x-x^2}{3} + \left( \frac{2x-x^2}{3}\right)^2 + \left( \frac{2x-x^2}{3}\right)^3\right]\]
Here, we have a lot less contributions from each term to the ones that we need (i.e. the constant, 1st power x, 2nd power x, 3rd power x 
terms all have a finite amount of terms contributing to their value). The first four nonzero terms are:
\[\frac23 \left[ 1 + \frac{2x}{3} + \frac{x^2}{9} - \frac{4x^3}{27}\right]\]
\subsection{Ratio Test}
nah
\end{document}
