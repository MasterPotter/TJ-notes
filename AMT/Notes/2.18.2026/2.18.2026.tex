\documentclass{article}
\usepackage[margin=1in]{geometry}
\usepackage{amsmath}
\usepackage{amssymb}
\usepackage{fancyhdr}


\pagestyle{fancy}
\lhead{Zoeb Izzi}
\rhead{2/10/2026}
\lfoot{Unit 1}
\rfoot{}
\begin{document}
\thispagestyle{fancy}
\begin{center}\LARGE{Advanced Mathematical Techniques (AMT) \\ Notes}\end{center}
\section{Today's stuff}
Let's try and prove that the alternating harmonic series:
\[1 - \frac12 + \frac13 - \frac14 + \frac15 - \frac16 + \cdots\]
converges.\newline \newline
We can start by going from 0 to 1 on a number line, then back to 1/2, then up
by 1/3, then down by 1/4, and so on until we see that the sequence is constrained
and converges between 0 and 1. \newline \newline

Further, we can see that if we choose to add the terms in a different 
order, such as
\[1 + \frac13 + \frac15 + \frac17 + \cdots + \frac1{2n+1} - \frac12 \cdots\]
we can get a different number. Therefore, this is \textbf{conditionally convergent},
where the order of the terms matters. We define the alternating series test
as a test where if we have a sequence of terms that alternate in signs and have
decreasing absolute value, we know the series converges.
\newline \newline
Let's try this complicated integral.
\[\int_{-\infty}^{\infty}\frac{x^4}{(x^2 + 4)^6}dx\]
Consider instead a simpler version:
\[\int_{-\infty}^{\infty} \frac{dx}{ax^2 + b} = \frac1a 
\int_{-\infty}^{\infty} \frac{dx}{x^2 + \frac{b}{a}}\]
\[ = \frac1a \frac2{\sqrt{\frac{b}{a}}}\lim_{R\rightarrow\infty} 
\arctan\left(\frac{x}{\sqrt{ \frac{b}{a}}}\right)\Big|_{0}^{R}\]
\[ = \frac{\pi}{\sqrt{ab}}\]
Returning to the original integral, we know that
\[\frac{\partial}{\partial a}\left(\frac{\pi}{\sqrt{ab}}\right) = -\frac{\pi}{2a^{3/2}b^{1/2}}\]
\[\frac{\partial^2}{\partial b^2}\left(\frac{\pi}{\sqrt{ab}}\right) = \frac{3\pi}{4a^{1/2}b^{5/2}}\]
Therefore, 
\[\frac{\partial^5}{\partial a^2 \partial b^3}\left(\frac{\pi}{\sqrt{ab}}\right) = - 2 \times 3 \times 4 \times 5
\times \int_{-\infty}^{\infty}\frac{x^4}{(x^2 + 4)^6}dx \]
\[ = \frac{}{}\]
Also, we'll be expected to know how to do Gaussian integrals.
\[I = \int_{-\infty}^{\infty}e^{-x^2}dx\]
Therefore, 
\[I^2 = \int_{-\infty}^{\infty}\int_{-\infty}^{\infty}e^{-(x^2+y^2)}dxdy\]
\[\int_{-\infty}^{\infty} dx \int_{-\infty}^{\infty} e^{-x^2-y^2}dy\]
Switching to polar,
\[\int_{0}^{\infty}rdr \int_{0}^{2\pi}d\theta e^{-r^2} = \pi\]
Therefore, 
\[I^2 = \pi\]
\[I = \sqrt{\pi}\]
Adding some sort of auxiliary paramter:
\[\int_{-\infty}^{\infty}e^{-ax^2}dx\]
We can just do a u-substitution to get that this is $\sqrt{\frac{\pi}{a}}$. We can also differentiate with respect to $a$ to get:
\[\int_{-\infty}^{\infty}e^{-ax^2}dx = \frac{1}{2}\sqrt{\pi}a^{-\frac12}\]
Another one with an auxiliary paramter:
\[\int_{0}^{1}x^ndx = \frac1{n+1}\]
Taking the partial with respect to $n$,
\[\frac{\partial}{\partial n}\left(\frac1{n+1}\right) = \int_{0}^{1}x^n \ln{x} dx = -\frac{1}{(n+1)^2}\]
\[\frac{\partial^2}{\partial n^2}\left(\frac1{n+1}\right) = \int_{0}^{1}x^n \ln^2{x} dx = \frac{2}{(n+1)^3}\]
\[\frac{\partial^3}{\partial n^3}\left(\frac1{n+1}\right) = \int_{0}^{1}x^n \ln^3{x} dx = -\frac{6}{(n+1)^4}\]
\newline \newline 
Moving on to something completely different, let's look at what's called an 
infinite product. Given the sequence $\{a_n\}$ of numbers, we can define 
$\{P_n\}_{n=1}^{\infty}$ via $P_1 = 1 + a_1$, $P_{n+1} = (1 + a_{n+1})P_n$. 
If \[\lim_{n\rightarrow \infty} P_n\] exists and is nonzero, then it is called
the infinite product:
\[\prod_{k=1}^{\infty}(1 + a_k)\]
Infinite products cannot converge to 0 - if they do we say they 
diverge. The reason is that if we allow that to happen, then any of the terms
could be 0 and that would rig your entire product. We make it $1 + a_k$ so that we 
can use some of the series stuff instead of having to reinvent everything. 
We can convert the product to the series by taking the logarithm:
\[\ln\left(\prod_{k=1}^{\infty}(1 + a_k)\right) = \sum_{k=1}^{\infty}\ln(1 + a_k)\]
Therefore, \[\prod_{k=1}^{\infty}(1 + a_k)\] converges if and only if 
\[\sum_{k=1}^{\infty}\ln(1 + a_k)\] converges. This turns out to be true for:
\[\frac12 |a_k| < \left\vert ln(1+a_k) \right\vert < 2|a_k|\]
This gives us a 3-way inequality, which means for absolute convergence, 
we have a way to relate sequences.
\end{document}
