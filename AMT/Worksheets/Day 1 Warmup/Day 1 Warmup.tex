\documentclass{article}
\usepackage{graphicx, physics, dsfont}
\usepackage{graphicx}
\usepackage{fancyhdr}
\usepackage{hyperref}
\usepackage{ragged2e}
\usepackage{amsmath, amsthm, amssymb}
\usepackage[margin=1in]{geometry}
\usepackage{setspace}
\usepackage{tikz}
\usetikzlibrary{positioning}
\usepackage[
]{geometry}
\pagestyle{fancy}
\lhead{Zoeb Izzi}
\rhead{2/5/2026}
\lfoot{Chapter 1}
\rfoot{}
\begin{document}
\thispagestyle{fancy}
\begin{center}\LARGE{Advanced Mathematical Techniques (AMT) \\ Worksheet 1}\end{center}
\vspace{10mm}
\begin{enumerate}
    \item Consider the sequence $\{ a_n\}^\infty_{n=1}$, where $a_n = \frac{(-1)^nn}{n+1}$
    \begin{enumerate}
        \item Write out the first 5 terms of the sequence. What general behavior can you deduce from these terms?
        \newline \newline The first 5 terms are:
        \begin{equation*}a_1=-\frac12, a_2=\frac23, a_3 =-\frac34, a_4=\frac45, a_5=-\frac56\end{equation*}
        From this, it looks like the absolute value of the sequence approaches 1, with each term's sign alternating.\newline
        \item Is this sequence monotonic? Is it bounded? Would you say that the sequence converges or diverges? If it 
        diverges, would you say that it contains a convergent subsequence? \newline \newline The sequence is 
        \textbf{not monotonic}, as shown above because $a_2>a_1$, but $a_3<a_2$. It \textbf{is bounded}, however, 
        because the absolute value of the sequence can never exceed 1, since $\frac{n}{n+1}$ can never exceed 1. I would say 
        \textbf{the sequence diverges}, since it will end up oscillating between numbers close to 1 and -1 for all 
        eternity, but it \textbf{does contain 2 convergent subsequences} - one of all the positive numbers, going to 1, 
        and one of all the negative ones, going to -1.\newline
        \item Determine the limit points of this sequence.\newline \newline
        
    \end{enumerate}
    \item Consider the sequence $\{ a_n\}^\infty_{n=1}$, where $a_1=3$ and $a_{n+1} = 7 - \frac{2}{a_n}$.
    \begin{enumerate}
        \item Write out the first 5 terms of the sequence. What general behavior of the terms do your results 
        suggest? Does it appear to be monotonic? Does it appear to be bounded?
        \newline \newline
        The first 5 terms are: 
        \begin{equation*}a_1 = 3, a_2 = \frac{19}3, a_3=\frac{127}{19}, a_4=\frac{851}{127}, a_5=\frac{5703}{851}\end{equation*}
        The terms seem to \textbf{converge} to a number just over 7. The sequence appears to be \textbf{monotonic} and bounded, 
        below by 3 and above by 7.\newline
        \item Show that this sequence is bounded, $ 2< a_n <7 $ for all $n$, by induction (assume that $2 < a_k<7$ for some value 
        of $k$, then use the recurrence relation to show that $2 < a_{k+1}<7$).
        \begin{equation*}2 < a_k < 7\end{equation*}
        \begin{equation*}\frac17 < \frac1{a_k} < \frac12\end{equation*}
        \begin{equation*}-1 < -\frac2{a_k}<-\frac27\end{equation*}
        \begin{equation*}6 < 7-\frac2{a_k}<\frac{47}7\end{equation*}
        \begin{equation*}6 < a_{k+1}<\frac{47}{7}\end{equation*}
        \begin{equation*}2 < 6 < a_{k+1} < \frac{47}7\end{equation*}
        \begin{equation*}\boxed{2 < a_{k+1}<\frac{47}7}\end{equation*}\newline
        \item The sequence is indeed monotonic increasing. Prove this by induction. Your proof will be similar 
        to that found in part (b).
        \newline \newline For the base case, where $n=1$:
        \begin{equation*}a_1 < a_2 \xrightarrow{} 3 < \frac{19}{3} \text{ is true.}\end{equation*}
        Next, let:
        \begin{equation*}a_k < a_{k+1}\end{equation*}
        \begin{equation*}\frac{1}{a_k} > \frac{1}{a_k+1}\end{equation*}
        \begin{equation*}-\frac{2}{a_k} < -\frac{2}{a_k+1}\end{equation*}
        \begin{equation*}7-\frac{2}{a_k} < 7-\frac{2}{a_k+1}\end{equation*}
        \begin{equation*}\boxed{a_{k+1} < a_{k+2}}\end{equation*}\newline
        \item Based on your analysis in (b) and (c), can you say for sure that this sequence converges? If so, 
        what number (exactly) does it converge to?
        \newline \newline 
        Since this sequence is bounded, it has at least one limit point, and since it's also monotonically increasing, it 
        \textbf{does converge.} It converges to:
        \begin{equation*}L = 7 - \frac2L\end{equation*}
        \begin{equation*}L^2 = 7L-2\end{equation*}
        \begin{equation*}L^2 - 7L + 2 = 0\end{equation*}
        \begin{equation*}L = \frac{7 \pm \sqrt{41}}{2}\end{equation*}
        Since $\frac{7-\sqrt{41}}2 < 3$, we know that the sequence converges to:
        \begin{equation*}\boxed{\frac{7 + \sqrt{41}}{2}}\end{equation*}
        \newline
        
    \end{enumerate}
    
    \item Consider the sequence $\{ a_n\}^\infty_{n=1}$, where $a_1=3$ and $a_{n+1} = 7 + \frac{2}{a_n}$.
    \begin{enumerate}
        \item Write out the first 5 terms of the sequence. What general behavior of the terms do your results suggest? 
        Does it appear to be monotonic? Does it appear to be bounded?
        \newline \newline
        The first 5 terms are: 
        \begin{equation*}a_1 = 3, a_2 = \frac{23}3, a_3=\frac{167}{23}, a_4=\frac{1215}{167}, a_5=\frac{8839}{1215}\end{equation*}
        The terms seem to \textbf{converge} to a number just over 7. The sequence is \textbf{not monotonic}, 
        since $a_4 > a_3$, but $a_3 < a_2$. It does appear to be bounded, below by 3 and above by 12.\newline
        \item This sequence is not monotonic, but it is bounded. Show using induction that the sequence is bounded by 2 from 
        below and 12 from above. What convergence property can you state definitively from this result?
        \newline \begin{equation*} 2 < a_n < 12 \end{equation*}
        \begin{equation*} \frac{1}{12}< \frac1{a_n} < \frac12 \end{equation*}
        \begin{equation*} \frac{1}{6}< \frac2{a_n} < 1 \end{equation*}
        \begin{equation*} \frac{43}{6}< 7 + \frac2{a_n} < 8 \end{equation*}
        \begin{equation*} \frac{43}{6}< a_{n+1} < 8 \end{equation*}
        \begin{equation*} 2 < \frac{43}{6}< a_{n+1} < 8 < 12 \end{equation*}
        \begin{equation*}\boxed{2 < a_{n+1} < 12 }\end{equation*}
        We can't state any convergence property from this because again, the sequence isn't monotonic, so there could be 
        two limit points, and no convergence.\newline

        \item The sequence is convergent. Determine exactly what number it converges to and verify this result from 
        investigating a few terms in the sequence.
        \[L = 7 + \frac2L\]
        \[L^2-7L-2=0\]
        \[L=\frac{7 \pm \sqrt{57}}{2}\]
        Since $\frac{7 - \sqrt{57}}{2} < 0$, and we know our sequence is bounded by 2 and 12, we know that the converged value
        is:
        \[\boxed{\frac{7 +\sqrt{57}}{2}}\]\newline
    \end{enumerate}
    \item Consider the sequence $\{ S_n\}_{n=1}^\infty$, where $S_1 = 1$ and $S_{n+1} = S_n + \frac1{n+1}$.
    \begin{enumerate}
        \item Write out the first 5 terms of the sequence. Does it appear to be monotonic? Does it appear to be bounded?\newline\newline
        The first five terms are:
        \[S_1 = 1, S_2 = \frac32, S_3 = \frac{11}6, S_4 = \frac{25}{12}, S_5 = \frac{137}{60}\]\newline
        It appears to be monotonically increasing, and it appears to be bounded by 3.\newline
        \item The difference between adjacent terms in this sequence, $S_{n+1} - S_n = \frac{1}{n+1}$, clearly goes to 0 as 
        $n \xrightarrow{} \infty$. Does this imply the sequence is convergent? Why or why not?
        \newline \newline No. Convergence requires bounds, and even though each term gets closer and closer to the next, that doesn't mean 
        that the sequence is bounded.\newline
        \item Prove that this sequence is monotonic (it is not difficult, and does not require induction). Does this mean it converges?
        \newline \newline \[n > 0, \frac{1}{n+1} > 0 \hspace{1mm}\forall \hspace{1mm}n, \hspace{1mm}\therefore\hspace{1mm} S_n \text{ is monotonically increasing.}\]
        Again, this does not necessarily mean that it converges, because we don't know that there are bounds on the sequence,
        so it could just increase forever, like how $1,2,3$ is monotonically increasing but obviously divergent.\newline
        \item Work through the logic of the following expression:
        \[S_n = 1 + \frac12 + \frac13 + \frac 14 (\text{last 2 sum to } > \frac12, < 1)...\]
        Can you state conclusively whether or not this sequence converges based on this analysis? If it converges, can you give
        a bound that its limit is definitely less than?\newline \newline I can say this doesn't converge because we know 
        that there are infinite of these "groups" in the limit. Each of these groups sums to a number greater than $\frac12$.
        Since there are infinite groups of them, then we know that there are infinite amounts of at least $\frac12$, meaning that
        the sum of this sequence will be greater than the sum of the sequence $\frac12, \frac12, \frac12, ... $, which diverges. Thus, 
        this sequence also diverges.
    \end{enumerate}
    \item Consider the sequence $\{ S_n\}_{n=1}^\infty$, where $S_1 = 1$ and $S_{n+1} = S_n + \frac1{(n+1)^2}$.
    \begin{enumerate}
        \item Write out the first 5 terms of the sequence. Does it appear to be monotonic? Does it appear to be bounded?\newline\newline
        The first five terms are:
        \[S_1 = 1, S_2 = \frac54, S_3 = \frac{49}{36}, S_4 = \frac{205}{144}, S_5 = \frac{5269}{3600}\]\newline
        It appears to be monotonically increasing, and it appears to be bounded below by 1 and above by 2.\newline
        \item The difference between adjacent terms in this sequence, $S_{n+1} - S_n = \frac{1}{(n+1)^2}$, clearly goes to 0 as 
        $n \xrightarrow{} \infty$. Does this imply the sequence is convergent? Why or why not?
        \newline \newline No. Convergence requires bounds, and even though each term gets closer and closer to the next, that doesn't mean 
        that the sequence is bounded.\newline
        \item Prove that this sequence is monotonic. Does this mean it converges?
        \newline \newline \[n > 0, \frac{1}{n+1} > 0, \frac1{n+1} > 0, \hspace{1mm}\forall \hspace{1mm}n, \hspace{1mm}\therefore\hspace{1mm} S_n \text{ is monotonically increasing.}\]
        Again, this does not necessarily mean that it converges, because we don't know that there are bounds on the sequence,
        so it could just increase forever, like how $1,2,3$ is monotonically increasing but obviously divergent. We don't know there
        is a limit point yet.\newline
        \item Work through the logic of the following expression:
        \[S_n = 1 + \frac1{2}^2 + \frac1{3}^2 + \frac 1{4^2} (\text{last 2 sum to } > \frac12, < 1)...\]
        Can you state conclusively whether or not this sequence converges based on this analysis? If it converges, can you give
        a bound that its limit is definitely less than?\newline \newline I can say this does converge because we know 
        that there are infinite of these "groups" in the limit. Each of these groups sums to a number less than $\frac12^p$, where $p$
        represents the "group" number. Since there are infinite groups of them, then we know that there are infinite amounts 
        of at most $\frac12^p$, meaning that the sum of this sequence will be less than the sum of the sequence $1, \frac12, \frac14, \frac18, ... $, which converges. Thus, 
        this sequence also converges.
    \end{enumerate}
    \item Consider the sequence $\{a_n \}_{n=1}^\infty$, where $a_n = cos(n^2)$.
    \begin{enumerate}
        \item Write out the first 5 terms of the sequence (decimal approximations are appropriate). Does it appear to be 
        monotonic? Does it appear to be bounded?\newline \newline

        The first 5 terms are:
        \[a_1 = 0.5403, a_2 = - 0.6536, a_3 = -0.9111, a_4 = -0.9577, a_5 = 0.9912\]
        This is not monotonic, since $S_1 > S_2$ but $S_5 > S_4$. It is bounded by -1 and 1 as lower and upper bounds,
        since those are the bounds of the cosine function.\newline
        \item The difference between adjacent terms in this sequence, $a_{n+1}-a_n = cos((n+1)^2) - cos(n^2)$, clearly doesn't
        go to 0 as $n\xrightarrow{} \infty$. Does this mean that the sequence doesn't converge?
        \newline \newline Yes. The limit of the sequence must approach 0 in order for the it to converge.\newline
        \item Prove that the sequence is bounded. What does this imply about its convergence characteristics?
        \newline \newline The sequence is bounded because of the inherent bounds on the cosine function, which are 
        $-1 \le \cos(n) \le 1 \hspace{1mm} \forall n$.\newline 
        \item Can you prove that this sequence has at least 1 limit point? Can you find a limit point of this sequence? 
        What is the difference between these two statements?\newline \newline
        The sequence has at least one limit point because it's an infinite sequence that's bounded, but we don't know what
        that limit point is because finding the limit points is far more difficult than ascertaining that one exists.
        The difference between the two is that one is just proving that something is there, while the other asks 
        what it is.
    \end{enumerate}
\end{enumerate}
\end{document}
