\documentclass{article}
\usepackage{graphicx, physics, dsfont}
\usepackage{fancyhdr}
\usepackage{hyperref}
\usepackage{ragged2e}
\usepackage{amsmath, amsthm, amssymb}
\usepackage[margin=1in]{geometry}
\usepackage{setspace}
\usepackage{tikz}
\usetikzlibrary{positioning}

\pagestyle{fancy}
\lhead{Zoeb Izzi}
\rhead{\today}
\lfoot{Problem Set 2}
\rfoot{}

\begin{document}
\thispagestyle{fancy}
\begin{center}\LARGE{Advanced Mathematical Techniques \\ Problem Set 2}\end{center}
\vspace{10mm}

\begin{enumerate}
    \item Find the first four nonzero terms in the Maclaurin series for $\frac{x^3}{e^{2x^2} - 1}$. What is the radius of
    convergence of this expansion?
    \newline \newline 
    \[\frac{x^3}{e^{2x^2} - 1} = \frac{x^3}{2x^2 + \frac{(2x^2)^2}{2!} + \frac{(2x^2)^3}{3!} + \frac{(2x^2)^4}{4!} + \cdots} = \frac{x^3}{2x^2 + 2x^4 + \frac{4x^6}{3} + \frac{2x^8}{3} + \frac{4x^{10}}{15} + \cdots}\]
    \[ = \frac{x^3}{2x^2(1 + x^2 + \frac{2x^4}{3} + \frac{x^6}{3} + \frac{2x^8}{15} + \cdots)} = \frac{x}{2(1 + x^2 + \frac{2x^4}{3} + \frac{x^6}{3} + \frac{2x^8}{15} + \cdots)}\]
    \[ = \frac{x}{2} \cdot \frac{1}{1 + x^2 + \frac{2x^4}{3} + \frac{x^6}{3} + \frac{2x^8}{15} + \cdots} = \frac{x}{2} \cdot \frac{1}{1 - \left(-x^2 - \frac{2x^4}{3} - \frac{x^6}{3} - \frac{2x^8}{15} - \cdots\right)}\]
    \[ = \frac{x}{2} \left( 1 + \left(-x^2 - \frac{2x^4}{3} - \frac{x^6}{3} - \frac{2x^8}{15} - \cdots\right) + \left(-x^2 - \frac{2x^4}{3} - \frac{x^6}{3} - \frac{2x^8}{15} - \cdots\right)^2 + \cdots\right)\]
    \[ = \boxed{\frac{x}{2} - \frac{x^3}{2} + \frac{x^5}{6} - \frac{x^9}{90}} + \cdots\]
    Radius of convergence will occur when $e^{2x^2} = 1$, or when $2x^2 = 0$. When $x = 0$, this is a removable singularity,
    so we consider the complex numbers. When $2x^2 = 2n\pi i$, we have a singularity, so the radius of convergence is 
    $\boxed{\sqrt{\pi}}$.\newline
    \item Find the first three nonzero terms in the Maclaurin series for $\tan(2x^3)$. What is the radius of convergence
    of this expansion? Use your result to determine the value of the 9th and 12th derivative of this function at zero.
    \newline \newline
    \[\tan(2x^3) = 2x^3 + \frac{(2x^3)^3}{3} + \frac{2(2x^3)^5}{15} + \cdots = \boxed{2x^3 + \frac{8x^9}{3} + \frac{64x^{15}}{15}} + \cdots\]
    Further, we know that the radius of convergence will occur when:
    \[|2x^3| < \frac{\pi}{2}\]
    \[|x| < \sqrt[3]{\frac{\pi}{4}}\]
    The radius of convergence is $\boxed{\sqrt[3]{\frac{\pi}{4}}}$.
    Also, from the expansion of the series, we know that it has an $x^9$ term and no $x^{12}$ term, which means the derivatives
    are nonzero and zero, respectively. 
    At $x=0$, the 9th derivative is $\boxed{9! \cdot \frac{8}{3}}$, and the 12th derivative is 
    $\boxed{0}$.\newline
    \item Determine the value of the integral $\int_{-\infty}^{\infty} \frac{x^2}{(x^2 + 9)^3} dx$
    \newline \newline
    \[\int_{-\infty}^{\infty} \frac{x^2}{(x^2 + 9)^3} dx = 2\int_{0}^{\infty} \frac{x^2}{(x^2 + 9)^3} dx\]
    \[x = 3\tan \theta, dx = 3 \sec^2 \theta d\theta\]
    \[2\int_{0}^{\infty} \frac{x^2}{(x^2 + 9)^3} dx = 2\int_{0}^{\frac{\pi}{2}} \frac{(9\tan^2\theta)}{(9\tan^2 \theta + 9)^3} \cdot 3 \sec^2 \theta d\theta = 2\int_{0}^{\frac{\pi}2} \frac{27 \tan^2 \theta \sec^2 \theta}{(9 \sec^2 \theta)^3} d\theta\]
    \[ = \frac{54}{729} \int_{0}^{\frac{\pi}{2}} \frac{\tan^2 \theta}{\sec^4 \theta} d\theta = \frac{2}{27} \int_{0}^{\frac{\pi}{2}} \sin^2 \theta \cos^2 \theta d\theta = \frac{2}{27} \cdot \frac{\pi}{16} = \boxed{\frac{\pi}{216}}\]
    \newline 
    \item Determine the value of the integral $\int_{-\infty}^{\infty} \frac{x^4}{(x^2+1)^6}dx$
    \newline \newline
    \[\int_{-\infty}^{\infty} \frac{x^4}{(x^2+1)^6} dx = 2 \int_{0}^{\infty}\frac{x^4}{(x^2+1)^6} dx\]
    \[x = \tan \theta, dx = \sec^2 \theta d\theta\]
    \[2 \int_{0}^{\infty}\frac{x^4}{(x^2+1)^6} dx = 2 \int_{0}^{\frac{\pi}{2}} \frac{\tan^4 \theta}{\sec^{10}\theta}d\theta\]
    \[ = 2 \int_{0}^{\frac{\pi}{2}} \sin^4 \theta \cos^6 \theta d\theta = 2 \int_{0}^{\frac{\pi}{2}} \left( \frac{1-\cos 2 \theta}{2}\right)^2 \left( \frac{1 + \cos 2\theta}{2}\right)^3 = 2 \int_{0}^{\frac{\pi}{2}} \frac{1}{32} (1-\cos 2\theta)^2 (1 + \cos 2\theta)^3 d\theta\]
    \[ = \frac{1}{16} \int_{0}^{\frac{\pi}{2}} (1 + \cos 2 \theta - 2 \cos^2 2 \theta - 2 \cos^3 2 \theta + \cos^4 2 \theta + \cos^5 2 \theta)d\theta = \boxed{\frac{3\pi}{256}}\]
    \newline
    \item Determine the value of the integral $\int_{0}^{1} x^7 \ln^4 x dx$
    \newline \newline
    \[u = -\ln x, x = e^{-u}, dx = -e^{-u}du\]
    \[\int_{0}^{1} x^7 \ln^4 x dx = \int_{\infty}^{0} (e^{-u})^7 u^4 (-e^{-u})du = \int_{0}^{\infty} e^{-8u} u^4 du\]
    \[t = 8u, dt = 8du, du = \frac{1}{8}dt\]
    \[\int_{0}^{\infty} e^{-8u} u^4 du = \int_{0}^{\infty} e^{-t} \left(\frac{t}{8}\right)^4 \cdot \frac{1}{8} dt = \frac{1}{8^5} \int_{0}^{\infty} e^{-t} t^4 dt = \frac{1}{8^5} \cdot 4! = \boxed{\frac{3}{4096}}\]
    \newline
    \item Determine the value of the integral $\int_{0}^{1} x\sqrt x \ln^2 x dx$
    \newline \newline
    \[\int_{0}^{1} x\sqrt x \ln^2 x dx = \int_{0}^{1} x^{3/2} \ln^2 x dx\]
    \[u = -\ln x, x = e^{-u}, dx = -e^{-u}du\]
    \[\int_{0}^{1} x^{3/2} \ln^2 x dx = \int_{\infty}^{0} (e^{-u})^{3/2} u^2 (-e^{-u})du = \int_{0}^{\infty} e^{-5u/2} u^2 du\]
    \[t = \frac{5u}{2}, dt = \frac{5}{2}du, du = \frac{2}{5}dt\]
    \[\int_{0}^{\infty} e^{-5u/2} u^2 du = \int_{0}^{\infty} e^{-t} \left(\frac{2t}{5}\right)^2 \cdot \frac{2}{5} dt = \frac{8}{125}\int_{0}^{\infty} e^{-t} t^2 dt = \frac{8}{125}\cdot 2! = \boxed{\frac{16}{125}}\]
    \newline
    \item Determine the value of the integral $\int_{0}^{\infty} x^6e^{-3x^2}$
    \newline \newline
    \[u = 3x^2, du = 6x dx, x^2 = \frac{u}{3}\]
    \[\int_{0}^{\infty} x^6e^{-3x^2} dx = \int_{0}^{\infty} = \int_{0}^{\infty} \left( \sqrt{\frac{u}{3}}\right)^6 \cdot e^{-u} \cdot \left( \frac{\sqrt 3}{6 \sqrt u}\right)du = \frac{\sqrt 3}{162}\int_{0}^{\infty} u^{\frac{5}{2}}\cdot e^{-u}\]
    \[= \frac{\sqrt 3}{162} \cdot \Gamma\left(\frac{7}{2}\right) = \frac{\sqrt 3}{162} \cdot \frac{7}{2} \cdot \frac{5}{2} \cdot \frac{3}{2} \cdot \Gamma\left(\frac 1 2 \right) = \boxed{\frac{5\sqrt{3\pi}}{432}}\]
    \newline
    \item The infinite product expansion for the sine function, $\frac{\sin x}{x} = \prod_{n=1}^{\infty} \left(1 - \frac{x^2}{n^2\pi^2}\right)$, 
    can be used to determine the value of many interesting infinite products.
    \begin{enumerate}
        \item Determine the value of $\prod_{k=1}^{\infty} \frac{4k^2-1}{4k^2}$
        \newline \newline
        \[\prod_{k=1}^{\infty} \frac{4k^2-1}{4k^2} = \prod_{k=1}^{\infty} \left(1 - \frac{1}{4k^2}\right) = 
        \prod_{k=1}^{\infty} \left(1 - \frac{1}{(2k)^2}\right)\]
        Here, $x^2 = \frac{\pi^2}{4}$, $x = \pm \frac{\pi}{2}$. For both of them, we see that \[\frac{\sin(x/2)}{x/2} = \boxed{\frac{2}{\pi}}\]
        \item Determine the value of $\prod_{k=1}^{\infty} \frac{9k^2-1}{9k^2}$
        \newline \newline 
        \[\prod_{k=1}^{\infty} \frac{9k^2-1}{9k^2} = \prod_{k=1}^{\infty} \left(1 - \frac{1}{9k^2}\right)\]
        Here, $x^2 = \frac{\pi^2}{9}$, $x = \pm \frac{\pi}{3}$. For both of them, we see that:
        \[\frac{\sin(x/3)}{x/3} = \boxed{\frac{3\sqrt3}{2\pi}}\]
        \item Explain why the infinite product $\prod_{k=1}^{\infty} \frac{k^2-1}{k^2}$ is obviously 0. Then, use a limit process
        with the sine function's infinite product to determine $\prod_{k=2}^{\infty} \frac{k^2-1}{k^2}$.
        \newline \newline
        It's obviously 0 since the first term is $\frac{0}{1} = 0$, and anything multiplied by 0 is 0. We can evaluate considering 
        the limit as $x$ approaches $\pi$:
        \[\frac{\sin x}{x} = \left( 1 - \frac{x^2}{\pi^2}\right) \cdot \prod_{k=2}^{\infty} \left(1 - \frac{x^2}{k^2\pi^2}\right)\]
        \[\prod_{k=2}^{\infty} \left(1 - \frac{x^2}{k^2\pi^2}\right) = \lim_{x\to\pi} \frac{\sin x}{x\left(1 - \frac{x^2}{\pi^2}\right)} = \lim_{x\to \pi} \frac{\pi^2\sin x}{x(\pi-x)(\pi + x)}\]
        \[ = \lim_{x \to \pi} \left[\frac{\pi^2}{x(\pi + x)} \cdot \frac{\sin x}{\pi - x}\right] \xrightarrow{\text{L'H\^opital's rule}} \frac{\pi^2}{\pi \cdot 2\pi} = \boxed{\frac{1}{2}}\]
        \item Use complex numbers along with the infinite product representation of the sine function to determine the value of $\prod_{k=2}^{\infty} \frac{k^2+1}{k^2}$
        \newline \newline
        \[\frac{k^2+1}{k^2} = 1 + \frac{1}{k^2} = 1 - \frac{(i\pi)^2}{k^2\pi^2} = 1 - \frac{x^2}{k^2\pi^2}, x = \pm i\pi\]
        \[\prod_{k=2}^{\infty} \frac{k^2+1}{k^2} = \frac{\sin x}{2x}\Big|_{x=\pm i\pi} = \boxed{\frac{\sin(i\pi)}{2i\pi}}\]
    \end{enumerate}
    \item Use complex numbers along with the infinite product representation of the sine function to determine the value of $\prod_{k=1}^{\infty} \frac{4k^2+1}{4k^2}$
    \newline \newline
    \[\frac{4k^2+1}{4k^2} = 1 + \frac{1}{4k^2} = 1 - \frac{(i\pi/2)^2}{k^2\pi^2} = 1 - \frac{x^2}{k^2\pi^2}, x = \pm i\frac{\pi}{2}\]
    \[\frac{\sin x}{x} = \frac{\sin(i\frac{\pi}{2})}{i\frac{\pi}{2}} = \boxed{\frac{2\sin(i\frac{\pi}{2})}{i\pi}}\]
\end{enumerate}
\end{document}