\documentclass[11pt]{article}
\usepackage[margin=1in]{geometry}
\usepackage{amsmath, amssymb}
\usepackage{physics} % For Dirac notation \ket{} and \bra{}
\usepackage{tikz}
\usetikzlibrary{quantikz} % Standard library for drawing quantum circuits

\title{Electrodynamics \\ Worksheet PS13 \\ \large (Quantum Computing)}
\author{Zoeb Izzi}
\date{February 9, 2026}

\begin{document}

\maketitle
\thispagestyle{empty}

\section*{1. Hadamard Transform}
Apply $H \otimes H \otimes H$ to $\ket{000}$ and show that the resulting state is a uniform superposition of all binary strings 
of length 3. If you measure the qubits, what possible outcomes can you get, and with what probabilities?\newline \newline

The operator generated by $H \otimes H \otimes H$ is as follows:
\[H \otimes H \otimes H = H \otimes (H \otimes H) = \frac1{\sqrt2}\begin{pmatrix} 1 & 1 \\ 1 & -1 \end{pmatrix} \otimes 
\frac12\left[\begin{pmatrix} 1 & 1 \\ 1 & -1 \end{pmatrix} \otimes \begin{pmatrix} 1 & 1 \\ 1 & -1 \end{pmatrix} \right]\]
\[= \frac{1}{2\sqrt2}\left[\begin{pmatrix} 1 & 1 \\ 1 & -1 \end{pmatrix} \otimes \begin{pmatrix} 1 & 1 & 1 & 1 \\ 
    1 & -1 & 1 & -1 \\ 1 & 1 & -1 & -1 \\ 1 & -1 & -1 & 1 \end{pmatrix}\right]\]
\[ = \frac1{2\sqrt2} \begin{pmatrix} 1 & 1 & 1 & 1 & 1 & 1 & 1 & 1 \\ 1 & -1 & 1 & -1 & 1 & -1 & 1 & -1 \\ 1 & 1 & -1 & -1 & 1 
& 1 & -1 & -1 \\ 1 & -1 & -1 & 1 & 1 & -1 & -1 & 1 \\ 1 & 1 & 1 & 1 & -1 & -1 & -1 & -1 \\ 1 & -1 & 1 & -1 & -1 & 1 & -1 & 1 
\\ 1 & 1 & -1 & -1 & -1 & -1 & 1 & 1 \\ 1 & -1 & -1 & 1 & -1 & 1 & 1 & -1 \end{pmatrix}\]
Applying this on our operator, $\ket{000}$, 
\[\frac1{2\sqrt2}\begin{pmatrix} 1 & 1 & 1 & 1 & 1 & 1 & 1 & 1 \\ 1 & -1 & 1 & -1 & 1 & -1 & 1 & -1 \\ 1 & 1 & -1 & -1 & 1 
& 1 & -1 & -1 \\ 1 & -1 & -1 & 1 & 1 & -1 & -1 & 1 \\ 1 & 1 & 1 & 1 & -1 & -1 & -1 & -1 \\ 1 & -1 & 1 & -1 & -1 & 1 & -1 & 1 
\\ 1 & 1 & -1 & -1 & -1 & -1 & 1 & 1 \\ 1 & -1 & -1 & 1 & -1 & 1 & 1 & -1 \end{pmatrix} \times \begin{pmatrix}
    1\\0\\0\\0\\0\\0\\0\\0
\end{pmatrix} = \frac1{\sqrt8}  \begin{pmatrix} 1\\1\\1\\1\\1\\1\\1\\1\end{pmatrix}\]
Therefore, there are 8 possible outcomes, each of which is one of the permuatations of 3 0s and 1s, and the probability
of getting any one of them is $\frac18$.

\section*{2. Some Gate Algebra}
Prove the following circuit identities. You can do this by showing the matrix representations of each side of the equality 
are the same.

\vspace{0.5cm}

\begin{enumerate}
    \item[\textbf{a}] $CNOT(X \otimes I) = (X \otimes X)CNOT$
    \begin{center}
        \begin{quantikz}
            \lstick{} & \gate{X} & \ctrl{1} & \qw \\
            \lstick{} & \qw & \targ{} & \qw 
        \end{quantikz}
        \quad = \quad
        \begin{quantikz}
            \lstick{} & \ctrl{1} & \gate{X} & \qw \\
            \lstick{} & \targ{} & \gate{X} & \qw 
        \end{quantikz}
    \end{center}
    \[\begin{pmatrix} 1 & 0 & 0 & 0 \\ 0 & 1 & 0 & 0 \\ 0 & 0 & 0 & 1 \\ 0 & 0 & 1 & 0 \end{pmatrix} \times 
    \left[\begin{pmatrix} 0 & 1 \\ 1 & 0 \end{pmatrix} \otimes \begin{pmatrix} 1 & 0 \\ 0 & 1 \end{pmatrix}\right]\]
    \[ = \begin{pmatrix} 1 & 0 & 0 & 0 \\ 0 & 1 & 0 & 0 \\ 0 & 0 & 0 & 1 \\ 0 & 0 & 1 & 0 \end{pmatrix} \times 
    \begin{pmatrix} 0 & 0 & 1 & 0 \\ 0 & 0 & 0 & 1 \\ 1 & 0 & 0 & 0 \\ 0 & 1 & 0 & 0 \end{pmatrix} = 
    \begin{pmatrix} 0 & 0 & 0 & 1 \\ 0 & 0 & 1 & 0 \\ 1 & 0 & 0 & 0 \\ 0 & 1 & 0 & 0 \end{pmatrix} = LHS\]
    Further, 
    \[X \otimes X = \begin{pmatrix} 0 & 0 & 0 & 1 \\ 0 & 0 & 1 & 0 \\ 0 & 1 & 0 & 0 \\ 1 & 0 & 0 & 0 \end{pmatrix}\]
    \[\begin{pmatrix} 1 & 0 & 0 & 0 \\ 0 & 1 & 0 & 0 \\ 0 & 0 & 0 & 1 \\ 0 & 0 & 1 & 0 \end{pmatrix}\times 
    \begin{pmatrix} 0 & 0 & 0 & 1 \\ 0 & 0 & 1 & 0 \\ 0 & 1 & 0 & 0 \\ 1 & 0 & 0 & 0 \end{pmatrix} = 
    \begin{pmatrix} 0 & 0 & 0 & 1 \\ 0 & 0 & 1 & 0 \\ 1 & 0 & 0 & 0 \\ 0 & 1 & 0 & 0 \end{pmatrix} = RHS = LHS\hspace{1mm}
     \checkmark\]

    \item[\textbf{b}] $CNOT(I \otimes X) = (I \otimes X)CNOT$
    \begin{center}
        \begin{quantikz}
            \lstick{} & \qw & \ctrl{1} & \qw \\
            \lstick{} & \gate{X} & \targ{} & \qw 
        \end{quantikz}
        \quad = \quad
        \begin{quantikz}
            \lstick{} & \ctrl{1} & \qw & \qw \\
            \lstick{} & \targ{} & \gate{X} & \qw 
        \end{quantikz}
    \end{center}

    Tensors are commutative, so we've already proven the LHS. Proving the RHS, 
    \[\begin{pmatrix} 0 & 0 & 0 & 1 \\ 0 & 0 & 1 & 0 \\ 0 & 1 & 0 & 0 \\ 1 & 0 & 0 & 0 \end{pmatrix}\times
    \begin{pmatrix} 1 & 0 & 0 & 0 \\ 0 & 1 & 0 & 0 \\ 0 & 0 & 0 & 1 \\ 0 & 0 & 1 & 0 \end{pmatrix}
    = \begin{pmatrix} 0 & 0 & 0 & 1 \\ 0 & 0 & 1 & 0 \\ 1 & 0 & 0 & 0 \\ 0 & 1 & 0 & 0 \end{pmatrix} = RHS = LHS\hspace{1mm} 
    \checkmark\]

    \item[\textbf{c}] $CNOT(I \otimes Z) = (Z \otimes Z)CNOT$
    \begin{center}
        \begin{quantikz}
            \lstick{} & \qw & \ctrl{1} & \qw \\
            \lstick{} & \gate{Z} & \targ{} & \qw 
        \end{quantikz}
        \quad = \quad
        \begin{quantikz}
            \lstick{} & \ctrl{1} & \gate{Z} & \qw \\
            \lstick{} & \targ{} & \gate{Z} & \qw 
        \end{quantikz}
    \end{center}
\end{enumerate}
\[\begin{pmatrix} 1 & 0 & 0 & 0 \\ 0 & 1 & 0 & 0 \\ 0 & 0 & 0 & 1 \\ 0 & 0 & 1 & 0 \end{pmatrix} 
\begin{pmatrix} 1 & 0 & 0 & 0 \\ 0 & -1 & 0 & 0 \\ 0 & 0 & 1 & 0 \\ 0 & 0 & 0 & -1 \end{pmatrix} = 
\begin{pmatrix} 1 & 0 & 0 & 0 \\ 0 & -1 & 0 & 0 \\ 0 & 0 & -1 & 0 \\ 0 & 0 & 0 & 1 \end{pmatrix} 
\begin{pmatrix} 1 & 0 & 0 & 0 \\ 0 & 1 & 0 & 0 \\ 0 & 0 & 0 & 1 \\ 0 & 0 & 1 & 0 \end{pmatrix}\]
\[\begin{pmatrix} 1 & 0 & 0 & 0 \\ 0 & -1 & 0 & 0 \\ 0 & 0 & 0 & -1 \\ 0 & 0 & 1 & 0 \end{pmatrix}
= \begin{pmatrix} 1 & 0 & 0 & 0 \\ 0 & -1 & 0 & 0 \\ 0 & 0 & 0 & -1 \\ 0 & 0 & 1 & 0 \end{pmatrix}\]
\[RHS = LHS \hspace{1mm} \checkmark\]

\section*{3. Building a Matrix Representation of a General Controlled Gate}
Similar to the CNOT gate, we can define a generalized controlled gate that applies a U gate to the second qubit if the first qubit is 1 following the rules:

\begin{align*}
    cU \ket{00} &= \ket{00} \\
    cU \ket{01} &= \ket{01} \\
    cU \ket{10} &= \ket{1} \otimes U \ket{0} \\
    cU \ket{11} &= \ket{1} \otimes U \ket{1}
\end{align*}

If we define the gate U as:
\begin{align*}
    U \ket{0} &= a \ket{0} + b \ket{1} \\
    U \ket{1} &= c \ket{0} + d \ket{1}
\end{align*}

\begin{enumerate}
    \item[\textbf{a}] By applying this definition of the U-gate to the controlled rules listed above, form the $4 \times 4$ matrix that corresponds to the cU gate.
    \[U = \begin{pmatrix}
        a & c \\ b & d
    \end{pmatrix}\]
    \[cU = \begin{pmatrix}
        1 & 0 & 0 & 0 \\
0 & 1 & 0 & 0 \\
0 & 0 & a & c \\
0 & 0 & b & d
    \end{pmatrix}\]
    
    \item[\textbf{b}] Take your generalization from part (a) and construct the controlled phase rotation gate if the regular phase rotation gate is given by:
    \[
    P(\phi) = \begin{pmatrix}
    1 & 0 \\
    0 & e^{i\phi}
    \end{pmatrix}
    \]
    \vspace{1mm}
    \[a = 1, b = 0, c = 0, d = e^{i\phi}\]
    \[\]
    \[cP(\phi) = \begin{pmatrix}
1 & 0 & 0 & 0 \\
0 & 1 & 0 & 0 \\
0 & 0 & 1 & 0 \\
0 & 0 & 0 & e^{i\phi}
\end{pmatrix} \]
\end{enumerate}

\end{document}