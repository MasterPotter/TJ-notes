\documentclass{article}
\usepackage{graphicx, physics, dsfont}
\usepackage{graphicx}
\usepackage{fancyhdr}
\usepackage{hyperref}
\usepackage{ragged2e}
\usepackage{amsmath, amsthm}
\usepackage[margin=1in]{geometry}
\usepackage{setspace}
\usepackage{tikz}
\usetikzlibrary{positioning}
\usepackage[
]{geometry}
\pagestyle{fancy}
\lhead{Zoeb Izzi}
\rhead{12/8/2025}
\lfoot{Chapter 3}
\rfoot{Problem Set 9}
\begin{document}
\thispagestyle{fancy}
\begin{center}\LARGE{Electrodynamics \\ Problem Set 9}\end{center}
\begin{center}\section*{Magnetic procession}\end{center}
\normalsize{Consider an electron that, at time $t = 0$ is in the state $\ket{\psi(t=0)} = \ket{+}_z = \frac 1 {\sqrt2} \left(\ket{+y} + \ket{-y}\right)$}\newline
\subsection*{\centering Part A}
Since the system hasn't been allowed to evolve at all, the state is just the initial, so the probabilities are:
\[\boxed{P(\ket{+x}) = \frac 12 \qquad P(\ket{-x}) = \frac 12}\]
Just as expected!
\begin{center}\subsection*{Part B}\end{center}

Following the 5 step process:\newline
\begin{itemize}
    \item Make the Hamiltonian.
\[\hat H = \frac{-q}{M_e}B_0 \frac \hbar 2 \begin{pmatrix} 0 & i \\ -i & 0 \end{pmatrix} = \frac{\omega \hbar}{2} \hat \sigma_y\]
    \item Determine the eigenstates and eigenenergies of $\hat H$.\newline\newline
    Defining $\omega = \frac{eB_0}m$,
        \[E_1 = \frac {\omega \hbar } 2, E_2 = -\frac{\omega \hbar} 2 \]
    \[\ket {E_1} = \ket{+y}, \ket{E_2} = \ket{-y}\]
    \item Write $\ket{\psi(0)}$ into the energy basis.\newline\newline
    Since $\ket{E_{1,2}} = \ket{\pm y}$,
    \[\ket{\psi_0} = (\bra{+y}\ket{\psi_0})\ket{+y} + \bra{-y}\ket{\psi_0})\ket{-y}\]
    \[\ket{\psi_0} = \frac1 {\sqrt2} \left( \ket{+y} + \ket{-y}\right)\]
    \[\ket{\psi_0} = \frac1 {\sqrt2} \left( \ket{E_1} + \ket{E_2}\right)\]
    \item Write down the time-evolved state
    \[\ket{\psi(t)} = \frac1 {\sqrt2} \ket{E_1} + \frac1{\sqrt2}e^{i\omega t}\ket{E_2}\]
    \item Make measurements! (in this case unnecessary)
\end{itemize}
Thus, the time-evolved state is just:
\[\boxed{\ket{\psi(t)} = \frac1 {\sqrt2} \ket{E_1} + \frac1{\sqrt2}e^{i\omega t}\ket{E_2}}\]
\begin{center}\subsection*{Part C}\end{center}
Reforming $\ket\psi$, 
\[\ket{\psi(t)} = \frac{1}{\sqrt2} \left[ \frac 1 {\sqrt2} \ket{+z} + \frac i {\sqrt2} \ket{-z}\right] + \frac{1}{\sqrt2} \left[ \frac 1 {\sqrt2} \ket{+z} - \frac i {\sqrt2} \ket{-z}\right]e^{i\omega t}\]
\[\left( \frac 1 2 + \frac 1 2 e^{i \omega t} \right)\ket{+z} + i \left( \frac 1 2 - \frac 1 2 e^{i\omega t}\right) \ket{-z}\]
Now, finding the actual probability:
\[P_{+x} = \left| \braket{+x}{\psi(t)} \right|^2\]
\[ = \left| \left(\frac 1 {\sqrt2} \ket{+z} + \frac1{\sqrt2}\ket{-z} \right)\ket{\psi(t)}\right|^2\]
\[ = \frac 1 8 \left| (1+e^{i\omega t}) + i(1-e^{i\omega t})\right|^2\]
\[ = \boxed{\frac 12 \left( 1 + \sin \omega t\right)}\]
Thus the probability of spin down is:
\[1 - \text(above) = \boxed{\frac 1 2 (1 - \sin \omega t)}\]

\section*{ \centering Three state system}
\normalsize{Let a Hamiltonian be defined by 
\[\hat H = \begin{pmatrix}
    A & 0 & B \\ 0 & C & 0 \\ B & 0 & A
\end{pmatrix}\] where $A,B,C$ are real numbers. Use the following notation for unit vectors:
\[\ket 1 = \begin{pmatrix}
    1\\0\\0
\end{pmatrix}, \ket 2 = \begin{pmatrix}
    0\\1\\0
\end{pmatrix}, \ket 3 = \begin{pmatrix}
    0\\0\\1
\end{pmatrix}\]}
\begin{center}\subsection*{Part A}\end{center}
By mass algebra that I don't want to type, we get:
\[\boxed{E_1 = A + B, E_2 = C, E_3 = A - B}\]
\[\boxed{\ket{E_1}= \frac 1{\sqrt2} \begin{pmatrix}
    1 \\ 0 \\ 1
\end{pmatrix}, \ket{E_2}= \begin{pmatrix}
    0 \\ 1 \\ 0
\end{pmatrix}, \ket{E_3}= \frac 1{\sqrt2} \begin{pmatrix}
    1 \\ 0 \\ -1
\end{pmatrix}}\]
\subsection*{\centering Part B}
This probability is encapsulated by:
\[\left|\braket{2}{\psi(t)}\right|^2\]
Let:
\[E_1 = \ket1, E_2 = \ket2, E_3 = \ket3\]
Therefore, since we start in state $\ket 2$,
\[\ket{\psi(t)} = e^{-\frac i \hbar Ct}\ket2\]
The probability of measuring $\ket2$ must therefore be:
\[\boxed{\left|\braket{2}{\psi(t)}\right|^2 = 1}\]
\begin{center}\subsection*{Part C}\end{center}
This probability is encapsulated by:
\[\left|\braket{3}{\psi(t)}\right|^2\]
But here,
\[\ket{\psi(t)} = \frac 1 {\sqrt2} \left( e^{-\frac i\hbar(A+B)t} \frac 1 {\sqrt2}(\ket1 + \ket3) + e^{-\frac i\hbar(A-B)t} \frac 1 {\sqrt2}(\ket1 - \ket3)\right)\]
Where, if we take our measurement:
\[\left|\braket{3}{\psi(t)}\right|^2 = \left| \frac 1 2 \left( e^{-\frac i \hbar (A+B)t} + e^{-\frac i \hbar (A-B)t}\right)\right|^2\]
Factoring out and simplifying:
\[\left|\braket{3}{\psi(t)}\right|^2 = \left| e^{-\frac i \hbar At} \cos\frac{Bt}{\hbar}\right|^2\]
Thus, 
\[\boxed{\left|\braket{3}{\psi(t)}\right|^2 = \cos^2\frac{Bt}{\hbar}}\]
\newline\newline
All done!

\end{document}
