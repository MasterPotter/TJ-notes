\documentclass{article}
\usepackage{graphicx, physics, dsfont}
\usepackage{fancyhdr}
\usepackage{hyperref}
\usepackage{ragged2e}
\usepackage{amsmath, amsthm}
\usepackage[margin=1in]{geometry}
\usepackage{setspace}
\usepackage{tikz}
\usetikzlibrary{positioning}
\pagestyle{fancy}
\lhead{Zoeb Izzi}
\rhead{2/10/2026}
\lfoot{Chapter 4.5}
\rfoot{}
\begin{document}
\thispagestyle{fancy}
\begin{center}\LARGE{Electrodynamics \\ Notes}\end{center}
\section{Recap/Intro}
Remember that the reason that quantum computers are better than classical computers
is that for given states, a quantum computer doesn't have to go through all of them - 
rather, it can just set up a bunch of states as a superposition and deal with it all in 
one fell swoop. This concept is called quantum parallelism, since quantum computers are 
essentially a bunch of classical computers operating side by side.\newline \newline
solution.
\section{Today's stuff}
Classically, we write code to find solutions. In quantum, we can write code to confirm a 
solution. For example, take the following two problems:
\[\text{Find all x for which } x^2 + x - 6 = 0\]
\[\text{Confirm that 2 is a solution to } x^2 + x - 6 = 0\]
Classically, doing the first is really bad - you'd have to just plug in over
and over. However, quantumly, just plug things in and they work out on their own by 
parallelism. As far as code flow goes, we can first generate all possible
solutions to a problem, and then pass that into an operator that confirms the valid 
among those "solutions".\newline \newline 
Essentially, that means that there are 3 steps to most quantum coding:
\begin{enumerate}
    \item Generate all possible solutions
    \item Confirm out which ones are valid via a parallelism structure
    \item Measure the results and done!
\end{enumerate}
Now, let's say Alice and Bob are playing a game. Here's what happens:
\begin{enumerate}
    \item Alice picks a random binary 4-bit number (e.g. 0000,0001,...1111) and 
    denotes it as a vector, $\overrightarrow{x}$. She gives that vector to Bob.
    \item Bob applies a function $f(\vec{x}) = \{ 0,1\}$ The function is 
    either constant (i.e. always returns 0 or 1) or it's a balance function 
    (by some parameters it selects whether to return a 0 or 1, each with 50\%
    probability).
    \item Bob returns that output to Alice.
    \item Alice tries to guess whether or not the function is balanced or constant.
\end{enumerate}
There's two possible constant functions, and two possible balanced functions, for N=1
qubits. 
\section{What's next}
\end{document}
