\documentclass{article}
\usepackage{graphicx, physics, dsfont, braket}
\usepackage{fancyhdr}
\usepackage{hyperref}
\usepackage{ragged2e}
\usepackage{amsmath, amsthm}
\usepackage[margin=1in]{geometry}
\usepackage{setspace}
\usepackage{tikz}
\usetikzlibrary{positioning}
\pagestyle{fancy}
\lhead{Zoeb Izzi}
\rhead{2/12/2026}
\lfoot{Chapter 4.5}
\rfoot{}
\begin{document}
\thispagestyle{fancy}
\begin{center}\LARGE{Electrodynamics \\ Notes}\end{center}
\section{Recap/Intro}
\section{Today's stuff}
Grover's Algorithm.
\begin{enumerate}
    \item Make a uniform superposition of all states. Do a Hadamard Transform on all input states.
    \item Feed into the oracle, which picks an answer. The oracle identifies the right state with a phase.
    \item Amplify the amplitude so that the right answer is pushed to have higher probability.
\end{enumerate}
For example, in a 2 qubit example, we'd start with (after the Hadamard transform):
\[\frac{1}{2}\left[\ket{00} + \ket{01} + \ket{10} + \ket{11}\right]\]
After applying the phase oracle, we see that it's changed the state to:
\[\frac{1}{2}\left[\ket{00} + \ket{01} + \ket{10} - \ket{11}\right]\]
Now, we just apply the amplification, which flips the state about the 
average so much so that the final output state is the only one that you have 
any realistic probability of getting.
\section{What's next}
\end{document}